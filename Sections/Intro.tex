\section{Introduction}
Quelques définitions:

\textbf{Télécommunication} : Transmission d'information sous la forme de signaux électrique sur un canal de communication.

\textbf{Signal} : Évolution de la tension en fonction du temps

	\subsection{Canaux de communication}
		2 type de canaux:
		\begin{itemize}
			\item \textbf{Filaire} : Ligne téléphonique, câbles coaxiaux, fibre optique, \dots
			\item \textbf{Sans fils} : Onde électromagnétique dans l'air ou espace
		\end{itemize}
		
		3 éléments les caractérise :
		\begin{itemize}
			\item \textbf{Atténuation} : Diminution de l'amplitude ou de la puissance d'un signal lors de sa transmission. Il augmente avec la distance
			\item \textbf{Bruit/Interférence} : Partie du signal ou on ne peut pas tirer de l'information.
			\item  \textbf{Distorsion/Dispersion} : Ensemble des modifications indésirable d'un signal. Il existe plusieurs source
			\begin{itemize}
				\item \textbf{Multitrajet} : Soit le signal arrive directement a la source, soit il rebondit sur des obstacles, le signal est alors découpé en plusieurs morceau de moindre intensité mais répartie sur le temps.
				\item \textbf{Effet Doppler} : La fréquence des signaux qui s'approche de nous est  différente de ceux qui s'éloigne de nous
			\end{itemize}
			
			\begin{figure}[H]
				\centering
				\includegraphics[width=\textwidth]{img/Distortion.png}
			\end{figure}
		\end{itemize}
	
	
	L'information est représentée sous la forme d'un signal. Cela veut dire que changer l'amplitude du signal ne change pas l'information. Mais la distorsion elle change l'information car le signal change de forme et donc modification du signal et de l'information.
	
	3 moyens de représenté le signal
	\begin{itemize}
		\item \textbf{Signaux analogiques}
		\item \textbf{Signaux numériques}
		\item \textbf{Numérisation} : Transformation de signaux analogique en numérique
	\end{itemize}
		\begin{figure}[H]
			\centering
			\includegraphics[width=0.6\textwidth]{img/Analogique-Numérique.png}
		\end{figure}
		
	\subsection{Transformation de Fourrier}
		Moyen de représenté un signal par un ensemble de fréquence. Elle permet d'analyser un signal en une somme de sinusoïde = \textbf{Contenue fréquentiel}
		\begin{figure}[H]
			\centering
			\includegraphics[width=0.6\textwidth]{img/Fourrier.png}
			\includegraphics[width=0.6\textwidth]{img/FourrierExemple.png}
		\end{figure}
		
		On peut voire sur les schéma que la fonction $f(t)$ est la somme des 2 sinusoide dans l'exemple donnée. Mais la fonction $F(\omega)$, si on ne regarde que du coté positif, $F(\omega)$ représente que il y a une sinusoide a 10kHz (Premiere fleches vers le haut) et un sinusoide a 30kHz, celle si pointe vers la bas car elle commence a un nombre négatif sur la schéma.
		
	\subsection{Bande de Fréquence}
		Tous les système de télécom sont limités en fréquence = la bande de fréquence du système.
		
		La \textbf{Modulation} permet de transposer un signal autour d'un fréquence définie. Utiliser pour partager les différentes bande de fréquence
		
		Un signal \textbf{modulé} est un signal \textbf{modulant} mis sur une \textbf{porteuse} à la fréquence désirée
		
		Le \textbf{Multiplexage} permet de faire passer plusieurs informations sur un seul support de fréquence. Les différents symboles sont combinés grâce a un multiplexeur. Deux types:
		\begin{itemize}
			\item \textbf{Temporel}
			\item \textbf{Fréquentiel}
		\end{itemize}
		\begin{figure}[H]
			\centering
			\includegraphics[width=0.6\textwidth]{img/Multiplexage.png}
		\end{figure}
	\subsection{Exemples}
		\subsubsection{Son}
			Un micro capture les ondes de pressions du à la voix sous la forme d'un signal électrique
			
			Le haut-parleur lui reçoit le signal électrique et fait vibrer une membrane pour recréer le son
			
			
		\subsubsection{Stéréo}
		
			Les sons gauches (G) et droites (D) sont envoyés ensemble sous un signal $S_1 = G+D$ donc la somme des 2. Grâce à ça les radios Mono peuvent recevoir le signal aussi. Un autre signal est aussi envoyé par la stéréo : $S_2 = G-D$. Le signal stéréo peut alors être recréé avec l'opération suivante


			\begin{align*} 
				G &=  S_1 + S2 = (G+D) + (G-D) = 2G \\ 		
				D &=  S_1 - S2 = (G+D) - (G-D) = 2D
			\end{align*}
			
			On va avoir un effet de compression car $S_2$ est souvent tres petit par rapport a $S_1$. $S_2$ a un contenu fréquentielle faible et aura donc d'un bande de fréquence plus faible
		
		\subsubsection{TV analogique}
		
			On joue sur la luminance (niveau de gris) des lignes est envoyé par pulsation de synchronisation (Synchronizing pulse) entre chaque ligne.

			Un canon à électrons envoie la lumière ligne après ligne sur l'écran pour afficher une image
			
\begin{figure}[H]
\centering
\begin{minipage}{.5\textwidth}
  \centering
  \includegraphics[width=.5\textwidth]{img/TV1.png}
\end{minipage}%
\begin{minipage}{.5\textwidth}
  \centering
  \includegraphics[width=.5\textwidth]{img/TV2.png}
\end{minipage}
\end{figure}
			
			Pour les couleurs, on fait pareil que la stéréo, on garde l'envoi de luminance pour que les veilles tv fonctionne encore.
			
			On a donc \textit{RGB} (red,green,blue) et \textit{YUV} les signaux (luminance Y et chrominance U V)
			
			\begin{align*} 
				Y &= 0.3\textcolor{red}{R} + 0.59\textcolor{green}{G} + 0.11\textcolor{blue}{B}\\
				U &= 0.493(\textcolor{blue}{B}-Y)\\
				V &= 0.877(\textcolor{red}{R}-Y)
			\end{align*}
	\subsection{Signaux numérique}
		C'est simple, c'est du binaire.
		
		Si on a un signal analogique et que on veut en faire un numérique $\rightarrow$ Numérisation.
		
		Il peut y avoire des erreur dans ce que on a transmit et ce que on lit.
		
	\subsection{Numérisation}
		Passer d'un signal analogique a un signal numérique. On va devoir \textbf{Echantillonner} le signal (capture de valeur a intervalle fixe). Car le signal numérique a des avantages :
		\begin{itemize}
			\item Traitement et stockage de l'info
			\item Regénération du signal par des codes correcteurs et détecteurs d'erreur
		\end{itemize}
		
		Signal analogique représenté par des valeurs continue  ($\mathbb{R}$)	et signal numérique par des valeurs discrète (0 ou 1).
		
		Le procédé de transformation appelé \textbf{Quantification}, on remplace les valeurs continue reçues par les valeurs discrète qui s'en rapproche le plus.
		
		Avantage du numérique est que le bruit dans l'analogique , si il est faible, sera filtré.
		
		Lors de l'échantillonnage, il faut pas prendre un fréquence trop grande sous peine d'avoir beaucoup d'erreur. Le \textbf{Théorème de Shannon} certifie que il n'y a aucune perte si la fréquence d'échantionnage est au moins 2 fois la fréquence maximum du signal
		
		\begin{figure}[H]
			\centering
			\includegraphics[width=0.6\textwidth]{img/Quantification.png}
		\end{figure}		
		
		
		